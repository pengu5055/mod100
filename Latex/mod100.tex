% This is a LaTeX template kindly taken from Jernej Debevec.
% Provided by Miha Muskinja for the purpose of the seminar I in the 1st year
% of the 2nd cycle of the study of physics at the Faculty of Mathematics and Physics, University of Ljubljana.

% Set the document class and options
\documentclass[10pt, titlepage, a4paper]{article}
\usepackage[a4paper, inner=2.5cm, outer=2.5cm, top=2.25cm, bottom=2.25cm]{geometry}
\usepackage{graphicx}
\usepackage{hyperref}
\usepackage{wrapfig}
\usepackage{amsmath}
\usepackage{amssymb}
\usepackage{amsfonts}
\usepackage{bm}
\hypersetup{colorlinks=true}

% Load the natbib package for citation style
\usepackage{natbib}

% Some macros for commonly used symbols in physics/quantum mechanics
\newcommand{\bb}[1]{\bm#1}
\newcommand{\dd}{\mathrm{d}}
\newcommand{\pp}{\partial}
\newcommand{\dg}{\dagger}
\newcommand{\la}{\langle}
\newcommand{\ra}{\rangle}
\newcommand{\id}{\mathbb{1}}
\newcommand{\T}{\mathsf{T}}
\newcommand{\ua}{\uparrow\>}
\newcommand{\da}{\downarrow\>}
\newcommand{\fs}[1]{\slashed{#1}}  % Feynmann slash
\newcommand{\mc}[1]{\mathcal{#1}}
\newcommand\thickbar[1]{\accentset{\rule{.5em}{.03em}}{#1}}
\renewcommand{\bar}{\thickbar}

% Start the document
\begin{document}

% The title page
\begin{titlepage}
{\centering
\includegraphics[width=6cm]{logo_fmf.pdf}

\vspace{0.8cm}
{\small Department of Physics}

\vspace{5cm}
\vspace{0.5cm}
{\huge\textbf{Data Flow in a Randomized Computer Network}} \\
\vspace{0.5cm}
{\large\textbf{Final Assignment for Model Analysis 1, 2023/24}}

\vfill
\textbf{Author:} Marko Urbanč \\
\textbf{Professor:} Prof. Dr. Simon Širca \\ 
\textbf{Advisor:}  doc. dr. Miha Mihovilovič \\

\vspace{1cm}
Ljubljana, August 2024 \\
}
\vspace{3cm}
\end{titlepage}

% Add table of conents
\hypersetup{pageanchor=true}
\pagenumbering{roman}
\setcounter{page}{2}
\tableofcontents
\vspace{1cm}

% Proceed with the main body
\pagenumbering{arabic}

% Sections based on typical Model Analysis structure
\section{Introduction}
The scope of computer networks has been expanding rapidly in the past few decades. What once were simple 
networks of interconnected computers have now evolved into complex systems that are used for a wide range of
applications. It makes sense then for one to plan and analyze the flow of data in such networks to ensure that
they are efficient and reliable and thus cheaper to maintain. \\

As my final assignment for Model Analysis 1, I opted to simulate the flow of data in a randomized computer network. The network 
consists of a $N\times N$ grid, where each unit of the grid can be a server, a user or a wire used to connect the two. Each of these 
units is connected to its four nearest neighbors and has a value associated with it that specifies it's bandwidth. For wires this is the 
maximum data throughput, for servers it is the maximum data processing rate and for users it is the maximum data consumption rate. We'd like 
to find the distribution of server-loads for a grid of wires with random bandwidths. It is possible to study many different types of 
server and user placements but the most interesting one to us will be where we have the users and servers placed on the top and bottom 
rows of the grid, respectively. \\

We can solve for the flow rates through the network by solving a set of constrained linear equations. This field of study is
known as linear programming and is a powerful tool for solving optimization problems. We've already seen some linear programming 
use in the second task of this course \texttt{mod102} where we created a dietary plan from a set of given foods based on their costs 
and nutritional values. For the sake of completeness it makes sense to quickly go over the basics of linear programming before we
proceed with the simulation. Linear programming uses linear optimization to find the maximum or minimum of a linear function
subject to a set of linear constraints. This function is known as the objective function and is quite analogous to the cost 
function we've seen mentioned in the world of Machine Learning. The constraints are linear inequalities or equations that
define the feasible region of the problem. Mathematically formulated we can consider a cost function $f(x)$ and a set of 
constraints defined as:
%
\begin{gather*}
    f(x_1, x_2, \dots, x_n) = c_1x_1 + c_2x_2 + \dots + c_nx_n\>, \\
    a_{11}x_1 + a_{12}x_2 + \dots + a_{1n}x_n \leq b_1\>, \\
    a_{21}x_1 + a_{22}x_2 + \dots + a_{2n}x_n \leq b_2\>, \\
    \vdots \\
    a_{m1}x_1 + a_{m2}x_2 + \dots + a_{mn}x_n \leq b_m\>.
\end{gather*}
%
The goal is then to find the values of $x_1, x_2, \dots, x_n$ that maximize or minimize the cost function $f(x)$ while satisfying
the constraints. The feasible region is the set of all points that satisfy the constraints and the optimal solution is the point in
the feasible region that maximizes or minimizes the cost function. That is all that is necessary from a mathematical perspective. Of course 
the actual implementation of solvers for such problems are much more complex and involve a lot of optimization techniques but that is 
not the focus of this assignment. \\

\section{Task}
The original text of the assignment reads as follows:
\begin{quote}
    \centering
    \textbf{Razporejanje prenosa podatkov po omrežju:} Za model omrežja vzemi $N\times N$ kvadratno mrežo, vsak rob pa ima naključno 
    maksimalno hitrost povezave med $0$ in $1$. Vozlišča na zgornjem robu so internetni odjemalci, spodnji rob pa so strežniki. V notranjih
    vozliščih velja 1. Kirchhoffov zakon. S pomočjo linearnega programiranja določi, kolikšne hitrosti prenosa imajo strežniki in odjemalci, 
    ko je skupna hitrost prenosa največja. Ker gre za naključna omrežja, si oglej tudi statistično porazdelitev zanimivih količin.
\end{quote}

\section{Solution Overview}
\section{Results}
\section{Conclusion and Comments}

% Add references
% \newpage
% \bibliographystyle{unsrt}
% \bibliography{mod100}

% End document
\end{document}
